%%%%%%%%%%%%%%%%%%%%%%%%%%%%%%%%%%%%%%%%%%%%%%%%%%%%%%%%%%%%%%%%%%%%%%%%%%%%%%%%
%
%  Project: Igloo Theory and Case Studies
%
%  Module:  document/root.tex (Isabelle/HOL 2020)
%  
%  root file for generation of PDF document
%
%
%%%%%%%%%%%%%%%%%%%%%%%%%%%%%%%%%%%%%%%%%%%%%%%%%%%%%%%%%%%%%%%%%%%%%%%%%%%%%%%%

\documentclass[11pt,a4paper]{report}
\usepackage{isabelle,isabellesym}

% additional packages
\usepackage{graphicx}       % to display session graph
\usepackage{a4wide}

% have each section start on a fresh page
\renewcommand{\isamarkupsection}[1]{\newpage\section{#1}}

% this should be the last package used
\usepackage{pdfsetup}

% for special symbols 
\usepackage{amssymb}

% urls in roman style, theory text in math-similar italics
\urlstyle{rm}
\isabellestyle{it}


\begin{document}


\title{Igloo: Soundly Linking Compositional Refinement and
Separation Logic for Distributed Systems Verification\\
A formalization in Isabelle/HOL}
\author{Christoph Sprenger and Tobias Klenze\\
\footnotesize sprenger@inf.ethz.ch, tobias.klenze@inf.ethz.ch}

\maketitle

\tableofcontents

\newpage

This is a generated file containing all of our definitions, theorems and proofs that we formalized in Isabelle/HOL in a human-readable form. 
The section \emph{Matching theorems from paper to the formalization} correlates the theorem and lemmas from the paper with their corresponding Isabelle formalization.
In particular, the theory dependencies given in the figure on the next page are useful.
Nevertheless, the most convenient way of browsing the Isabelle theories is to use the Isabelle GUI. See the README for details.

% sane default for proof documents
\parindent 0pt\parskip 0.5ex

% display the theory dependency graph
%%%%%%%%%%%%%%%%%%%%%%%%%%%%%%%%%%%%%%%%%%%%%%%%%%%%%%%%%%%%%%%%%%%%%%%%%%%%%%%%
%
%  Project: Igloo Theory and Case Studies
%  
%  session graph for PDF document
%
%
%%%%%%%%%%%%%%%%%%%%%%%%%%%%%%%%%%%%%%%%%%%%%%%%%%%%%%%%%%%%%%%%%%%%%%%%%%%%%%%%

\begin{figure}[p]
  \begin{center}
    \includegraphics[scale=.4]{session_graph.pdf}
    \caption{Theory dependencies}
  \end{center}
  \label{fig:theorydependencies}
\end{figure}


% generated text of all theories
\input{session}

% optional bibliography
%\bibliographystyle{abbrv}
%\bibliography{root}

\end{document}

